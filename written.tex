\documentclass[11pt,largemargins]{homework}

\newcommand{\hwname}{Jake Zimmerman}
\newcommand{\hwemail}{jezimmer}
\newcommand{\hwtype}{Homework}
\newcommand{\hwnum}{1}
\newcommand{\hwclass}{15-131}
\newcommand{\hwlecture}{1}
\newcommand{\hwsection}{A}

\begin{document}
\maketitle

\question

% Let's put statement of problem in italic for the lulz
\textit{Prove that \(A \cup (B \cap C) = (A \cup B) \cap C\), where \(A\),
\(B\), and \(C\) are sets such that \(A \subseteq C\)}.

% This makes the fancy italic "Proof." show up in the document
\begin{proof}
  % You'll notice that I'm using \( ... \) to encode inline mathematical
  % statements. This is the same as $ ... $, but nicer because you can see which
  % parentheses are matched with each other.
  Let \(x \in A \cup (B \cap C)\) be arbitrary and fixed. Then:

  % Some notes:
  % You can use \begin{align*} ... \end{align*} to make centered, "display" math
  % You can use \tag{...} to provide justification for your steps
  % Indentation makes it easy to reason about your syntax if you have errors.
  \begin{align*}
    x \in A \cup (B \cap C) &\iff x \in A \vee x \in (B \cap C)                          \tag{Definition of \(\cup\)} \\
                            &\iff x \in A \vee (x \in B \wedge x \in C)                  \tag{Definition of \(\cap\)} \\
                            &\iff (x \in A \wedge x \in C) \vee (x \in B \wedge x \in C) \tag{\(A \subseteq C\)} \\
                            &\iff (x \in A \vee x \in B) \wedge x \in C                  \tag{Distributivity} \\
                            &\iff (x \in (A \cup B) \wedge x \in C                       \tag{Definition of \(\cup\)} \\
                            &\iff x \in (A \cup B) \cap C                                \tag{Definition of \(\cap\)} \\
  \end{align*}

  Therefore, since we have shown that an arbitrary element of \(A \cup (B \cap
  C)\) is in \(A \cup (B \cap C)\) and vice versa, we have proven that these two
  sets are equal.
\end{proof}

\question

% \[ ... \] is like \( ... \) except that it puts the math on its own line, like
% \begin{align*} ... \end{align*}
\textit{For any \(n \in \mathbb{N}\) define \(S_n = \sum_{i = 1}^{n}{i^2}\).
Prove \[\forall n \in N. \, S_n = \frac{n(n + 1)(2n + 1)}{6}\]}

\begin{proof}
  Let \(P(n) \iff S_n = \frac{n(n + 1)(2n + 1)}{6}\). We will prove \(P(n)\) by
  induction on \(n \in \mathbb{N}\).

  \begin{induction}
    \basecase
      When \(n = 0\), \(P(0) \iff \frac{(0)(1)(1)}{6} = S_0\). Since \(S_0 =
      0\), \(P(n)\) is true.
    \indhyp
      Assume \(P(k)\) for some \(k \in \mathbb{N}\).
    \indstep
      Note that \(S_k = \frac{k(k + 1)(2k + 1)}{6}\) and that
      \begin{align*}
        S_{k + 1} = \sum_{i = 1}^{k + 1}{i^2} = \sum_{i = 1}^{k}{i^2} + (k + 1)^2 = S_k + (k + 1)^2 \\
      \end{align*}
      Therefore, we can substitute and rewrite the expression as follows:
      \begin{align*}
        S_{k+1} &= S_k + (k + 1)^2 \\
                &= \frac{k(k + 1)(2k + 1)}{6} + (k + 1)^2 \\
                &= (k + 1)\left(\frac{k(2k + 1) + (k + 1)}{6}\right) \\
                &= \frac{(k + 1)}{6}\left(k(2k + 1) + 6(k + 1)\right) \\
                &= \frac{(k + 1)}{6}(2k^2 + 7k + 6) \\
                &= \frac{(k + 1)((k + 1) + 1)(2(k + 1) + 1)}{6} \\
      \end{align*}
      Thus we can conclude that \(P(k + 1)\) is true.
  \end{induction}
  Therefore, because the base case and the induction step hold, \(P(n)\) is true
  for all \(n \in \mathbb{N}\) by induction.
\end{proof}

% We can force a page break like this. It helps when otherwise you'd end up with
% a short bit of text disconnected from the rest on the previous page (try
% removing it and seeing how it looks without).
\newpage
\question

\begin{alphaparts}
  \questionpart
    \textit{Count the rectangles of all sizes and of all positions that are formed
    using segments in a grid with \(m\) horizontal and \(n\) vertical lines.}

    \textit{Solution.} A rectangle is uniquely described by two distinct vertical
    lines and two distinct horizontal lines. Therefore we can just select these two
    lines and multiply these quantities by rule of product:
    \begin{align*}
      \binom{n}{2}\binom{m}{2}
    \end{align*}

  \questionpart
    % Quotes in LaTeX are weird: use `` ... '' to get the proper curly quotes.
    \textit{How many ways are there to rearrange the letters in the word
    ``anagram''?}

    \textit{Solution.} We can choose an arrangement of the letters in
    ``anagram'' in two steps. We first choose 3 of the 7 positions to be a's,
    then permute ``ngrm'' in the remaining positions. Thus, we have
    \begin{align*}
      \binom{7}{3} 4!
    \end{align*}
    ways to choose an arrangement.
\end{alphaparts}


\end{document}
